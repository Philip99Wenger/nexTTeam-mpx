{\bfseries Edit a file, create a new file, and clone from Bitbucket in under 2 minutes}

When you\textquotesingle{}re done, you can delete the content in this R\+E\+A\+D\+ME and update the file with details for others getting started with your repository.

{\itshape We recommend that you open this R\+E\+A\+D\+ME in another tab as you perform the tasks below. You can \href{https://youtu.be/0ocf7u76WSo}{\tt watch our video} for a full demo of all the steps in this tutorial. Open the video in a new tab to avoid leaving Bitbucket.} 



\subsection*{Edit a file}

You’ll start by editing this R\+E\+A\+D\+ME file to learn how to edit a file in Bitbucket.


\begin{DoxyEnumerate}
\item Click {\bfseries Source} on the left side.
\item Click the R\+E\+A\+D\+M\+E.\+md link from the list of files.
\item Click the {\bfseries Edit} button.
\item Delete the following text\+: {\itshape Delete this line to make a change to the R\+E\+A\+D\+ME from Bitbucket.}
\item After making your change, click {\bfseries Commit} and then {\bfseries Commit} again in the dialog. The commit page will open and you’ll see the change you just made.
\item Go back to the {\bfseries Source} page. 


\end{DoxyEnumerate}

\subsection*{Create a file}

Next, you’ll add a new file to this repository.


\begin{DoxyEnumerate}
\item Click the {\bfseries New file} button at the top of the {\bfseries Source} page.
\item Give the file a filename of {\bfseries contributors.\+txt}.
\item Enter your name in the empty file space.
\item Click {\bfseries Commit} and then {\bfseries Commit} again in the dialog.
\item Go back to the {\bfseries Source} page.
\end{DoxyEnumerate}

Before you move on, go ahead and explore the repository. You\textquotesingle{}ve already seen the {\bfseries Source} page, but check out the {\bfseries Commits}, {\bfseries Branches}, and {\bfseries Settings} pages. 



\subsection*{Clone a repository}

Use these steps to clone from Source\+Tree, our client for using the repository command-\/line free. Cloning allows you to work on your files locally. If you don\textquotesingle{}t yet have Source\+Tree, \href{https://www.sourcetreeapp.com/}{\tt download and install first}. If you prefer to clone from the command line, see \href{https://confluence.atlassian.com/x/4whODQ}{\tt Clone a repository}.


\begin{DoxyEnumerate}
\item You’ll see the clone button under the {\bfseries Source} heading. Click that button.
\item Now click {\bfseries Check out in Source\+Tree}. You may need to create a Source\+Tree account or log in.
\item When you see the {\bfseries Clone New} dialog in Source\+Tree, update the destination path and name if you’d like to and then click {\bfseries Clone}.
\item Open the directory you just created to see your repository’s files.
\end{DoxyEnumerate}

Now that you\textquotesingle{}re more familiar with your Bitbucket repository, go ahead and add a new file locally. You can \href{https://confluence.atlassian.com/x/iqyBMg}{\tt push your change back to Bitbucket with Source\+Tree}, or you can \href{https://confluence.atlassian.com/x/8QhODQ}{\tt add, commit,} and \href{https://confluence.atlassian.com/x/NQ0zDQ}{\tt push from the command line}. 